\chapter{UART Communication Implementation}\label{uart_implementation}

In this appendix we present two implementations of UART communication. The first one is a busy-waiting implementation, while the second one is an interrupt-driven implementation. These implementations are specific for the board used\supercite{mega2560_datasheet}.
\section{UART Control Registers}

On the board used, the UART serial interface is controlled by the following set of registers.
\begin{itemize}
	\item UBRRn (UART Baud Rate Register) is used to control the frequency of the byte transmission.
	\item UCSRn (UART Control and Status Register) allows to configure other aspects of the port and to poll if the UART is ready to transmit data or has received some.
	\item UDRn (UART Data Register) is the data register containing the data being sent and received.
\end{itemize}


\section{UART Busy-Waiting Implementation}\label{uart_bw}

In this first section we show an easy busy-waiting implementation of UART communication. Before starting any operation in fact the CPU polls until the end of the previous transmission.
\begin{ccode}
	//baud rate at which we want the serial to communicate
	#define UART_BAUD_RATE 38400

	//baud rate = F_CPU / (16*(UBRR+1))
	#define UBRR ( (F_CPU/16)/UART_BAUD_RATE - 1 )

	/*
	 * initializes the UART at UART_BAUD_RATE baud rate
	 */
	void UART_Init(void){
		//sets baud rate
		UBRR0H = (uint8_t)((UBRR)>>8);
		UBRR0L = (uint8_t)UBRR;
	
		//enables receiver and transmitter
		UCSR0B = (1<<RXEN0)|(1<<TXEN0);
	
		//sets data packet format: 8data, 2stop bit
		UCSR0C = (1<<UCSZ01)|(1<<UCSZ00);
	}

	/*
	 * transmits a char through the UART module.
	 *		It waits till previous char is transmitted (i.e. until UDRE is set),
	 *		then the new byte to be transmitted is loaded into UDR.
	 */
	void UART_TxByte(uint8_t data) {
		//waits until the transmit buffer is empty
		while ( !( UCSR0A & (1<<UDRE0) ) );
	
		//puts data into buffer, sends the data
		UDR0 = data;
	}

	/*
	 * receives a char from the UART module.
	 *		It waits till a char is received (i.e.till RXC is set),
	 *		then returns the received char.
	 */
	uint8_t UART_RxByte(void) {
		//waits for the data to be received
		while ( !(UCSR0A & (1<<RXC0) ) );
	
		//gets and returns received data from buffer
		return UDR0;
	}
\end{ccode}


\section{UART Interrupt-Driven Implementation}

In the following section we will present an interrupt-driven implementation of UART communication. The UART module in fact offers interrupt support, enabling the application software to carry on other operations during UART byte transfers.\\

The UART module will fire a Receive Interrupt (RX) when a new byte has been received, and a Data Register Empty Interrupt when the transmit buffer is ready to receive new data.

\paragraph{UART Data Handling}
We have decided to store the data to be transmitted and the data received in two global buffers automatically populated by the ISRs.
\begin{ccode}
	#define UART_BUFFER_SIZE 255

	typedef struct UART {
		uint8_t tx_buffer[UART_BUFFER_SIZE];
		volatile uint8_t tx_start;
		volatile uint8_t tx_end;
		volatile uint16_t tx_size;

		uint8_t rx_buffer[UART_BUFFER_SIZE];
		volatile uint8_t rx_start;
		volatile uint8_t rx_end;
		volatile uint16_t rx_size;
		
		uint16_t baudrate;
	} UART;
\end{ccode}

\paragraph{UART Module Initialization}
The main difference between this and the previous (\ref{uart_bw}) UART initialization is that this time we must enable the generation of interrupts by the UART module.
\begin{ccode}
	// Baud Rate at which we want the serial to communicate
	#define UART_BAUD_RATE 57600

	// Baud Rate = F_CPU / (16*(UBRR+1))
	#define UBRR ( (F_CPU/16)/UART_BAUD_RATE - 1 )
	
	static UART uart0;

	static void UART_buffers_init(void) {
		//initialize rx_buffer
		uart0.rx_start = 0;
		uart0.rx_end = 0;
		uart0.rx_size = 0;
	
		//initialize tx_buffer
		uart0.tx_start = 0;
		uart0.tx_end = 0;
		uart0.tx_size = 0;
	}

	/*
	 * UART_Init :
	 * 	initializes the UART at UART_BAUD_RATE baud rate
	 */
	void UART_Init(void){
		//initializes the global buffers
		UART_buffers_init();

		uart0.baudrate = UART_BAUD_RATE;
	
		//sets baud rate
		UBRR0H = (uint8_t)((UBRR)>>8);
		UBRR0L = (uint8_t)UBRR;
	
		//enables receiver and transmitter
			// RXCIE0 : RX Complete Interrupt Enable. Set to allow receive complete interrupts.
		UCSR0B = (1<<RXEN0)|(1<<TXEN0)|(1<<RXCIE0);
		//sets frame format: 8data, 2stop bit
		UCSR0C = (1<<UCSZ01)|(1<<UCSZ00);
	
		sei();
	}
\end{ccode}

\paragraph{UART Exposed Functions}
The followings are the exposed functions the CPU can use to enqueue data to be transmitted or get data that has been received.
\begin{ccode}
	/*
	 * UART_TxByte :
	 *  puts a byte in the transmit buffer
	 *		then as soon as possible it will be transmitted
	 *			through the UART module
	 */
	void UART_TxByte(uint8_t data) {
		//loops until there is some space in the buffer
		while (uart0.tx_size >= UART_BUFFER_SIZE);

		ATOMIC_BLOCK(ATOMIC_RESTORESTATE){
		  uart0.tx_buffer[uart0.tx_end] = data;
		  uart0.tx_end++;
		  
		  //if we reached the end of the circular buffer,
		  	//we go back to the beginning
			if (uart0.tx_end >= UART_BUFFER_SIZE)
				uart0.tx_end = 0;
		
			uart0.tx_size++;
		}
		
		//activates buffer empty interrupt
		UCSR0B |= (1<<UDRIE0);
	}

	/*
	 * UART_RxByte :
	 *  gets a byte from the receive buffer
	 *		@return : the received byte.
	 */
	uint8_t UART_RxByte(void) {
		//loops until there is some data in the buffer
		while(uart0.rx_size == 0);
	
		uint8_t data;
		
		ATOMIC_BLOCK(ATOMIC_RESTORESTATE){
		  data = uart0.rx_buffer[uart0.rx_start];
			uart0.rx_start++;
		
		  //if we reached the end of the circular buffer,
		  	//we go back to the beginning
			if (uart0.rx_start >= UART_BUFFER_SIZE)
				uart0.rx_start = 0;
		
			uart0.rx_size--;
		}
		
		return data;
	}
\end{ccode}

\paragraph{Interrupt Service Routines}
The first ISR is fired upon receipt of a byte, while the second one is fired when the transmit buffer is ready to transmit new data.
\begin{ccode}
	//Receive Interrupt Handler
	ISR(USART0_RX_vect) {
		//reads the new byte from UDR0,
			//and writes it in rx_buffer
	
		if (uart0.rx_size < UART_BUFFER_SIZE) {
			uart0.rx_buffer[uart0.rx_end] = UDR0;
			uart0.rx_end++;
		
		  //if we reached the end of the circular buffer,
		  	//we go back to the beginning
			if (uart0.rx_end >= UART_BUFFER_SIZE)
				uart0.rx_end = 0;
	
			uart0.rx_size++;
		}
	}

	//Data Register Empty (UDRE0) Flag
		//indicates whether the transmit buffer
		//is ready to receive new data
	ISR(USART0_UDRE_vect) {	
		//if the transmit buffer is empty
			//disables buffer empty interrupts
		if( uart0.tx_size == 0 )
			UCSR0B &= ~(1<<UDRIE0);
		else {
			//gets a byte from the tx_buff
				//adn writes it on UDR0
			UDR0 = uart0.tx_buffer[uart0.tx_start];
			uart0.tx_start++;
	
			//if we reached the end of the circular buffer,
				//we go back to the beginning
			if (uart0.tx_start >= UART_BUFFER_SIZE)
				uart0.tx_start = 0;
	
			uart0.tx_size--;
		}
	}
\end{ccode}
