\chapter{Introduction}\label{intro_localization}

In a world where automation and artificial intelligence are getting everyday more important, one of the fundamental problems for an autonomous robot is to be able to "understand" the environment from the measurements of its sensors. An autonomously navigating robot should know where it stands in the environment and what the environment looks like.\\

In this thesis we focus on the problem of localization. Localization\supercite{localization} is the process of determining where a robot is located with respect to its environment, using the measurements up to the current instant. Localization is a fundamental skill for an autonomous robot as the knowledge of the robot's own location is an essential prerequisite to making decisions about future actions. Therefore localization is a starting point for mapping and path planning.\\
Usually a map of the environment is available and the robot is equipped with sensors that observe the environment as well as monitor its own motion in order to be able to estimate its position in the map. In our scenario the robot does not have knowledge of a map of the environment and it is only equipped with sensors measuring its motion; our goal is to be able to correctly estimate the robot's position in relation to its starting point.

